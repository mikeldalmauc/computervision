
% LaTeX Language and campus name and format package
\documentclass[es,gi]{ifirak}\usepackage[]{graphicx}\usepackage[]{color}
%% maxwidth is the original width if it is less than linewidth
%% otherwise use linewidth (to make sure the graphics do not exceed the margin)
\makeatletter
\def\maxwidth{ %
  \ifdim\Gin@nat@width>\linewidth
    \linewidth
  \else
    \Gin@nat@width
  \fi
}
\makeatother

\definecolor{fgcolor}{rgb}{0.345, 0.345, 0.345}
\newcommand{\hlnum}[1]{\textcolor[rgb]{0.686,0.059,0.569}{#1}}%
\newcommand{\hlstr}[1]{\textcolor[rgb]{0.192,0.494,0.8}{#1}}%
\newcommand{\hlcom}[1]{\textcolor[rgb]{0.678,0.584,0.686}{\textit{#1}}}%
\newcommand{\hlopt}[1]{\textcolor[rgb]{0,0,0}{#1}}%
\newcommand{\hlstd}[1]{\textcolor[rgb]{0.345,0.345,0.345}{#1}}%
\newcommand{\hlkwa}[1]{\textcolor[rgb]{0.161,0.373,0.58}{\textbf{#1}}}%
\newcommand{\hlkwb}[1]{\textcolor[rgb]{0.69,0.353,0.396}{#1}}%
\newcommand{\hlkwc}[1]{\textcolor[rgb]{0.333,0.667,0.333}{#1}}%
\newcommand{\hlkwd}[1]{\textcolor[rgb]{0.737,0.353,0.396}{\textbf{#1}}}%
\let\hlipl\hlkwb

\usepackage{framed}
\makeatletter
\newenvironment{kframe}{%
 \def\at@end@of@kframe{}%
 \ifinner\ifhmode%
  \def\at@end@of@kframe{\end{minipage}}%
  \begin{minipage}{\columnwidth}%
 \fi\fi%
 \def\FrameCommand##1{\hskip\@totalleftmargin \hskip-\fboxsep
 \colorbox{shadecolor}{##1}\hskip-\fboxsep
     % There is no \\@totalrightmargin, so:
     \hskip-\linewidth \hskip-\@totalleftmargin \hskip\columnwidth}%
 \MakeFramed {\advance\hsize-\width
   \@totalleftmargin\z@ \linewidth\hsize
   \@setminipage}}%
 {\par\unskip\endMakeFramed%
 \at@end@of@kframe}
\makeatother

\definecolor{shadecolor}{rgb}{.97, .97, .97}
\definecolor{messagecolor}{rgb}{0, 0, 0}
\definecolor{warningcolor}{rgb}{1, 0, 1}
\definecolor{errorcolor}{rgb}{1, 0, 0}
\newenvironment{knitrout}{}{} % an empty environment to be redefined in TeX

\usepackage{alltt}

% ERABILIKO DIREN PAKETEAK %

% listings pakage is for code formating
\usepackage{listings}
% Paquete for acents and other special characters
% It is not necesary to use all this packages add or remove those you are interested on
\usepackage[utf8]{inputenc}
\usepackage{colortbl}
\usepackage[table]{xcolor}
\usepackage{graphicx}
\usepackage{wrapfig}
\usepackage{amsfonts}
\usepackage{makeidx}
\usepackage{adjustbox}
\usepackage{booktabs}
\usepackage{amsmath}

% Definition of colors
\definecolor{darkgreen}{rgb}{0,0.5,0}
\definecolor{lightgray}{rgb}{0.95,0.95,0.95}
\definecolor{gray}{rgb}{0.85,0.85,0.85}
\definecolor{white}{rgb}{1,1,1}
\definecolor{purple}{rgb}{0.51,0,0.25}
\definecolor{orange}{rgb}{0.255,0.178,0.102}
\definecolor{mygreen}{RGB}{28,172,0} 
\definecolor{mylilas}{RGB}{170,55,241}

\lstset{language=Matlab,%
    %basicstyle=\color{red},
    breaklines=true,%
    morekeywords={matlab2tikz},
    keywordstyle=\color{blue},%
    morekeywords=[2]{1}, keywordstyle=[2]{\color{black}},
    identifierstyle=\color{black},%
    stringstyle=\color{mylilas},
    commentstyle=\color{mygreen},%
    showstringspaces=false,%without this there will be a symbol in the places where there is a space
    %numbers=left,%
    %numberstyle={\tiny \color{black}},% size of the numbers
    %numbersep=20pt, % this defines how far the numbers are from the text
    emph=[1]{for,end,break},emphstyle=[1]\color{red}, %some words to emphasise
    %emph=[2]{word1,word2}, emphstyle=[2]{style},    
    backgroundcolor=\color{lightgray},
}

\DeclareMathSizes{10}{10}{10}{10}

\graphicspath{imagenes}
\renewcommand{\contentsname}{Indice}
\IfFileExists{upquote.sty}{\usepackage{upquote}}{}
\begin{document}

% Course year
\ikasturtea{2018 - 2019}
% Subject or course name
\irakasgaia{Visión por Computador}
% Title
\title{Desafíos 5, 6 y 7}
% Name of Author
\author{Mikel Dalmau}

\maketitle



%\section{Código}

\tableofcontents

\pagebreak
\section{Enunciados}


\pagebreak
\subsection{Código completo desafío 7}
\begin{lstlisting}
imagen_validacion = imread('validacion.gif');
numRows = 1200;
numCols = 1200;

wavelengthMin = 4/sqrt(2);
wavelengthMax = hypot(numRows,numCols);
n = floor(log2(wavelengthMax/wavelengthMin));
wavelength = 2.^(0:(n-2)) * wavelengthMin;

deltaTheta = 45;
orientation = 0:deltaTheta:(180-deltaTheta);
g = gabor(wavelength, orientacion);
gabormag = imgaborfilt(imagen_validacion,g);

for i = 1:length(g)
    sigma = 0.5*g(i).Wavelength;
    K = 3;
    gabormag(:,:,i) = imgaussfilt(gabormag(:,:,i),K*sigma); 
end

% When constructing Gabor feature sets for classification, it is useful to add a map of spatial location information in both X and Y. This additional information allows the classifier to prefer groupings which are close together spatially.
X = 1:numCols;
Y = 1:numRows;
[X,Y] = meshgrid(X,Y);
featureSet = cat(3,gabormag,X);
featureSet = cat(3,featureSet,Y);

%Normalize the features to be zero mean, unit variance.
numPoints = numRows*numCols;
X = reshape(featureSet,numRows*numCols,[]);

%indices = kmeans(muestra,16);
X = bsxfun(@minus, X, mean(X));
X = bsxfun(@rdivide,X,std(X));


L = kmeans(X,16,'Replicates',5);

%% MOstrar resultado
L = reshape(L,[numRows numCols]);
figure
imshow(label2rgb(L))

\end{lstlisting}
\begin{thebibliography}{arauak}
	
	\bibitem[1]{key-1} Matlab documentación oficial:\textit{es.mathworks.com}
	
\end{thebibliography}



\end{document}
